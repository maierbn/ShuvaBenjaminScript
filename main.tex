\input{header.tex}

\graphicspath{
{images/}
}

\begin{document}

\setcounter{tocdepth}{2}

\section{Formulation of Hyperelasticity}

The derivation follows the FEniCS example at \url{http://www.karlin.mff.cuni.cz/~blechta/pub/fenics-tutorial-2015/elasticity/doc.html}
and the book of Holzapfel \cite{holzapfel2000nonlinear}.

We consider an incompressible hyperelastic continuum in a domain $Ω$, with material points $\bfX \in Ω$ for a time span $[0,T]$. 
We compute the unknown displacement $\bfU(\bfX,t) \in \R^d$ and 
deformation velocity $\bfV(\bfX,t) \in \R^d$ by the following equations.

\begin{equation}
  \begin{array}{rll}
    \dot{\bfU} &= \bfV & \text{in }Ω × [0,T],  \\[4mm]
    \rho_0\,\dot{\bfV} &= \Div(\bfP) + \bfB & \text{in }Ω × [0,T], \qquad(\Div(\bfP) = \p{P_{aA}}{X_A}) \\[4mm]
    p := J - 1 &= 0,& \text{in }Ω × [0,T], \qquad \text{Lagrange multiplier for incompressiblity}\\[4mm]
    \bfU = \bfV &= 0 & \text{on } Γ_D × [0,T],  \qquad \text{Dirichlet boundary condition}\\[4mm]
    \bfsigma\,\bfn &= \bft & \text{on } Γ_N × [0,T],  \qquad \text{Neumann boundary condition} \\[4mm]
    \bfsigma\,\bfn &= 0 &\text{on } Ω\backslash Γ_D \cup Γ_N × [0,T],  \qquad \text{Neumann boundary condition} \\[4mm]
    \bfU = \bfV &= 0 & \text{on }Ω×0\qquad \text{initial condition},
  \end{array}
\end{equation}
where 
\begin{equation}
  \begin{array}{rll}
    \bfB &= 0 \qquad & \text{constant body force} \\[4mm]
    \bfF &= \bfI + ∇_\bfX\bfU    \qquad &\text{deformation gradient} \\[4mm]
    J &= \det\bfF           \qquad &\text{material determinant} \\[4mm]
    \bfb &= \bfF\,\bfF^{\top} \qquad &\text{left Cauchy Green / Finger tensor}\\[4mm]
    \bfsigma &= -p\bfI + \mu\,(\bfb - \bfI), \qquad &\text{Cauchy stress}\\[4mm]
    \bfP &= J\,\bfsigma\,\bfF^{-\top},  \qquad &\text{PK1 stress} \\[4mm]
    \bfS &= \bfF^{-1}\,\bfP  \qquad & \text{PK2 stress (not needed here)} \\[4mm]
    \mu &= E/3, \quad E = 10^5.
  \end{array}
\end{equation}

\subsection{Variational Formulation}

Approximation of time derivative using Crank-Nicolson:
\begin{equation}
  \begin{array}{rll}
    \dot{\bfU} = f(\bfU) \quad\leadsto\quad \dfrac1{dt}(\bfU^{(i+1)} - \bfU^{(i)}) = (1-θ)\,f(\bfU^{(i)}) + θ\,f(\bfU^{(i+1)}) \\[4mm]
    \dot{\bfV} = f(\bfV) \quad\leadsto\quad \dfrac1{dt}(\bfV^{(i+1)} - \bfV^{(i)}) = (1-θ)\,f(\bfV^{(i)}) + θ\,f(\bfV^{(i+1)}) \\[4mm]
  \end{array}
\end{equation}
Formulation in weak form, first equation:
\begin{equation}
  \begin{array}{rll}
    \ds\int_{\Omega}\dot{\bfU}\cdot \bfphi_u \,\d\bfX &= \ds\int_{\Omega} \bfV\cdot \bfphi_u \,\d\bfX \\[4mm]
    \Leftrightarrow \quad \dfrac1{dt}\,\ds\int_{\Omega}(\bfU^{(i+1)} - \bfU^{(i)})\cdot \bfphi_u \,\d\bfX &= \ds\int_{\Omega} \big((1-θ)\,\bfV^{(i)} + θ\,\bfV^{(i+1)}\big)\cdot \bfphi_u \,\d\bfX \\[4mm]
  \end{array}
\end{equation}
Second equation:
\begin{equation}
  \begin{array}{rll}
    \rho_0\,\ds\int_{\Omega}\dot{\bfV}\cdot\bfphi_v \,\d\bfX &= \ds\int_{\Omega} \Div(\bfP)\cdot\bfphi_v\,\d\bfX \\[4mm]
  \Leftrightarrow \rho_0\,\ds\int_{\Omega}\dfrac1{dt}(\bfV^{(i+1)} - \bfV^{(i)})\cdot\bfphi_v \,\d\bfX 
  &= \ds\int_{\Omega} \Div( (1-θ)\,\bfP^{(i)} + θ\,\bfP^{(i+1)})\cdot\bfphi_v\,\d\bfX \\[4mm]
  &= -\ds\int_{\Omega} \big((1-θ)\,\bfP^{(i)} + θ\,\bfP^{(i+1)}\big)\,∇\bfphi_v\,\d\bfX \\[4mm]
  \end{array}
\end{equation}
Third equation:
\begin{equation}
  \begin{array}{rll}
    J - 1 &= 0 \\[4mm]
    \leadsto \quad \theta\,J^{(i+1)}+ (1-\theta)\,J^{(i)} - 1 &= 0 \\[4mm]
  \Leftrightarrow \ds\int_Ω \big(θ\, p^{(i+1)}+(1-θ)\, p^{(i)}\big)\,\bfphi_p\,\d\bfX &= 0
  \end{array}
\end{equation}
Neumann boundary condition: Transfer traction vector $\bft$ in current configuration to traction $\bfT$ in reference configuration using Nansons formula:
\begin{equation}
  \begin{array}{rll}
    \d\bfs = J\, \bfF^{-\top}\cdot \d\bfS
  \end{array}
\end{equation}
%We have $\bfT(\bfx)\,\d S = \bft\,\d s$
We have $\bft = t\,\d\bfs = t\,J\,\bfF^{-\top}\cdot \d\bfS$.

\begin{equation}
  \begin{array}{rll}
    \text{rhs} = \ds\int_{Γ_N} J\,\bfF^{-\top}\,\bft\cdot \bfphi_v\,\d\bfS
  \end{array}
\end{equation}
In summary the weak form is given by
\begin{equation}
  \begin{array}{rll}
   \bfT_1 + \bfT_2 + \bfT_3 - \bfT_4 = 0,
     \end{array}
\end{equation}
with
\begin{equation}
  \begin{array}{rll}
    \bfT_1 &= \ds\int_Ω \dfrac1{dt} (\bfU^{(i+1)} - \bfU^{(i)})\cdot \bfphi_u \,\d\bfX - \int_Ω \big(θ\,\bfV^{(i+1)} \cdot \bfphi_u + (1-θ)\,\bfV^{(i)}\cdot \bfphi_u\big)\,\d\bfX \\[4mm]
    \bfT_2 &= \ds\int_Ω \dfrac1{dt} (\bfV^{(i+1)} - \bfV^{(i)})\cdot \bfphi_v \,\d\bfX+\,\ds\int_Ω\big(  θ\,\bfP^{(i+1)}\,∇\bfphi_v + (1-θ)\,\bfP^{(i)}\,∇\bfphi_v\big) \,\d\bfX\\[4mm]
    \bfT_3 &= \ds\int_Ω \big(θ\, p^{(i+1)}+(1-θ)\, p^{(i)}\big)\,\bfphi_p\,\d\bfX\\[4mm]
    \bfT_4 &= \ds\int_{Γ_N} J\,\bfF^{-\top}\,\bft\cdot \bfphi_v\,\d\bfS
  \end{array}
\end{equation}
\subsection{Static problem}
For the static problem we assume
\begin{equation*}
  \begin{array}{lll}
    \bfV^{(i)} = \bfzero, \qquad \bfU^{(i+1)} = \bfU^{(i)}.
  \end{array}
\end{equation*}
We get the equations
\begin{equation*}
  \begin{array}{lll}
    \bfT_1 &= \bfzero\\[4mm]
    \bfT_2 &= \,\ds\int_Ω\big(  θ\,\bfP^{(i+1)}\,∇\bfphi_v + (1-θ)\,\bfP^{(i)}\,∇\bfphi_v\big) \,\d\bfX\\[4mm]
    \bfT_3 &= \ds\int_Ω \big(θ\, p^{(i+1)}+(1-θ)\, p^{(i)}\big)\,\bfphi_p\,\d\bfX\\[4mm]
    \bfT_4 &= \ds\int_{Γ_N} J\,\bfF^{-\top}\,\bft\cdot \bfphi_v\,\d\bfS,
  \end{array}
\end{equation*}
\begin{equation*}
  \begin{array}{lll}
    \bfT_2 + \bfT_3 = \bfT_4
  \end{array}
\end{equation*}

\nocite{*}
\bibliography{literature}{}
\bibliographystyle{abbrv}
\end{document}
